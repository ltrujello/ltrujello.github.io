\begin{center}
\begin{tikzpicture}[>=stealth]
\node at (-2.25, 2.7) {$\rr^3$};
\tdplotsetmaincoords{70}{120}
\begin{scope}[tdplot_main_coords]

\pgfmathsetmacro{\scale}{5} 
\coordinate (O) at (0, 0, 0);         
\coordinate (Y) at (0, \scale-1, 0);    
\coordinate (Z) at (0, 0, \scale-2);  
\coordinate (X) at (\scale, 0, 0);    



\draw[semithick, ->] (O) -- (Y) node[right] {$y$};
\draw[semithick, ->] (O) -- (Z) node[above] {$z$};
\draw[semithick, ->] (O) -- (X) node[below] {$x$};


\draw[-stealth, thick, red] (0,0,0) -- (0, 0, 2);
\draw[-stealth, thick, Green!80!black] (0, 0, 0) -- (2, 0, 0);
\draw[-stealth, thick, blue] (0, 0, 0) -- (0, 2, 0);

\end{scope}
\draw[->] (2,2) to[bend left] (5,2);

\begin{scope}[xshift = 9cm, yshift = 1cm]
\node at (-2.15, 1.8) {$P_n(x)$};
\filldraw[rounded corners, ProcessBlue!10]
(-3, -1.5) rectangle (3, 1.5);
\node at (0,0) {$\textcolor{red}{1}$\quad $\textcolor{Green!80!black}{x}$\quad $\textcolor{blue}{x^2}$};
\node at (-1.5, -0.8) {... $1 + x$ ...};
\node at (1.2, 0.8) {... $2x^2 + 2x + 1$ ...};
\node at (1.4, -1) {... $x^2 + 2x$ ...};
\node at (-1.7, 1.1) {... $x^2 + 5$ ...};
\end{scope}
\end{tikzpicture}

\emph{We can specify a linear transformation from $\mathbb{R}^3$ 
to the polynomial vector space $P_3(x)$ by specifying where we send the basis 
elements. Here, we color code where we send the basis. 
}
\end{center}