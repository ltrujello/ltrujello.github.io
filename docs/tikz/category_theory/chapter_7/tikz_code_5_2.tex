\begin{center}
        \begin{tikzpicture}
            \pic[local bounding box=my braid,braid/.cd, 
            number of strands = 3, 
            thick,
            name prefix=braid]
            {braid={ s_1, s_1 s_2}}; 
            \draw[thick] 
            ([xshift=-1ex]my braid.north west) --  ([xshift=1ex]my braid.north east)
            ([xshift=-1ex]my braid.south west) --  ([xshift=1ex]my braid.south east);
        \end{tikzpicture}        
        \raisebox{1.5cm}{$\otimes$}
        \begin{tikzpicture}
            \pic[local bounding box=my braid,braid/.cd, 
            number of strands = 3, 
            thick,
            name prefix=braid]
            {braid={s_2, s_1 s_2}}; 
            \draw[thick] 
            ([xshift=-1ex]my braid.north west) --  ([xshift=1ex]my braid.north east)
            ([xshift=-1ex]my braid.south west) --  ([xshift=1ex]my braid.south east);
        \end{tikzpicture}        
        \raisebox{1.5cm}{$=$}
        \begin{tikzpicture}
            \pic[local bounding box=my braid,braid/.cd, 
            number of strands = 6, 
            thick,
            name prefix=braid]
            {braid={s_1-s_5 s_1-s_4 s_2-s_5}}; 
            \draw[thick] 
            ([xshift=-1ex]my braid.north west) --  ([xshift=1ex]my braid.north east)
            ([xshift=-1ex]my braid.south west) --  ([xshift=1ex]my braid.south east);
        \end{tikzpicture}      
        
        \emph{The braids $\sigma_1\sigma_1\sigma_2$
        and $\sigma_2\sigma_1\sigma_2$ are summed together 
        to obtain the braid $\sigma_1\sigma_1\sigma_2\sigma_5\sigma_4\sigma_5$
        above on the right.}
    \end{center}